\subsection{Brushless Motoransteuerung}
    Dieses Kapitel wurde in Zusammenarbeit mit der Gruppe \DasAndereTeam erstellt. 
    \subsubsection{Theorie der Ansteuerung}
        \begin{wrapfigure}{r}{0.50\textwidth}
           	\includegraphics[scale=0.45]{\BrushlessPath/Bilder/ZeitlicheHallSensorAnsteuerung.jpg}
           	\caption[Zeitliche Darstellung der Ansteuerung mit Hall-Sensoren]{Zeitliche Darstellung der Ansteuerung mit Hall-Sensoren \cite{AppNote:BrushlessuC}}
           	\centering
            \label{abb:ZeitlicheAnsteuerungBrushlessMotor}
        \end{wrapfigure}
        Brushless-Motoren sind Synchron-Drehstrom-Motoren. Das heisst, sie werden mittels eines kontinuierlichen Drehfeld in Bewegung gesetzt. Dabei ist darauf zu achten, dass der Läufer dem Drehfeld synchron folgen kann, daher der Name. Falls der Läufer dem Drehfeld aus irgend einem Grund nicht folgen kann, so wird keine Spannung vom Rotor in die Statorwicklungen induziert die der Erregerspannung entgegenwirkt. Daraus Folgt, dass ein immenser Strom fliesst, der nur von der Wicklungsimpedanz des Motors begrenzt wird.\\
        \\
        Es gibt hauptsächlich drei Methoden das Drehfeld zu generieren und zu regeln. Die eine und einfache Methode ist die Zwangskommutierung. Dabei wird ein Drehfeld erzeugt und dem Motor aufgezwungen. Der Läufer muss dem Drehfeld folgen können. Dabei ist ein maximaler Winkel zwischen dem Feld und dem Läufer von 90$^\circ$  zulässig. Wird deser Winkel überschritten wird der Motor zum Stillstand kommen mit den erwähnten Folgen.\\
        \\
        Die zweite Methode zur Regelung ist mittels drei Hallsensoren, die im Motor integriert sind. Dies macht den Motor aufwändiger und dementsprechend teurer. Die Regelung mit Hallsensoren ist verhältnismässig einfach, da je nach den Signalen die einzelnen Spulen direkt angesteuert werden kann. Der Zusammenhang zwischen der Ansteuerung und den Hall-Sensorsignalen ist in Abbildung \ref{abb:ZeitlicheAnsteuerungBrushlessMotor} ersichtlich. Dabei stehen $U$, $V$ und $W$ für die Phasenströme und $H_1$, $H_2$ und $H_3$ die entsprechenden Signale der Hallsensoren. Dieser Darstellung ist zu entnehmen, dass jedesmal wenn ein Hallsensor eine Änderung anzeigt, ein Nulldurchgang im entsprechenden Stromverlauf stattgefunden hat. Dies ist der Zeitpunkt, in dem die Kommutierung durchgeführt werden muss.\\
        \\
        Die dritte Möglichkeit ist, indem man einen virtuellen Sternpunkt bildet und mittels Komparatoren die Sternpunktdurchgänge detektiert. In der Controller-Logik muss der der Zeitunterschied der Kommutierung bis zum durchschreiten des Sternpunktes gemessen werden. Diese Zeit muss nochmal abgewartet werden bevor die Kommutierung durchgeführt werden.
    
    \subsubsection{Neuer Ansatz}
        \begin{wrapfigure}{r}{0.40\textwidth}
           	\includegraphics[scale=0.46]{\BrushlessPath/Bilder/PrinzipDerRekonstruktion.png}
           	\centering
           	\caption[Schema des Rekonstruktionsprinzip]{Schema des Rekonstruktionsprinzip \cite{HSLU:Pluess}}
            \label{abb:PrinzipRekonstruktion}
        \end{wrapfigure}
        In einem modifizierten Ansatz wird versucht, ob die Hall-Sensorsignale aus den Ansteuerungen des Motors gewonnen werden kann. Hierzu wird eine Schaltung pro Phase benötigt, mit der man die Nulldurchgänge beim virtuellen Sternpunkt detektieren zu können. Die Abbildung \ref{abb:PrinzipRekonstruktion} zeit die Schaltung, mit der dies realisiert werden kann. Mit dem Flip-Flop kann die PWM aus dem Sensorsignal unterdrückt werden. Diese rekonstruierte Hall-Sensor-Signale können direkt logischen verknüpft und genutzt werden, um den Motor mittels einer Dreiphasen-H-Brücke anzusteuern \cite{HSLU:Pluess}. Anhand des zeitlichen Verlaufs, der in Abbildung \ref{abb:ZeitlicheAnsteuerungBrushlessMotor} zu entnehmen ist, und der Ansteuerung einer H-Brücke ergibt sich die Wahrheitstabelle, die in Abbildung \ref{abb:WahrheitstabelleAnsteuerung} abgebildet ist. Die Signale $U_h$ symbolisiert den Highside-Transistor der U-Phase auf der H-Brücke und die $U_l$ entspricht dem Lowside-Transistor.\\      
        
        \begin{figure}[h!]
            \begin{tabular}{ccc||cc|cc|cc||c}
                 $H_1$ & $H_2$ & $H_3$ & $U_h$ & $U_l$ & $V_h$ & $V_l$ & $W_h$ & $W_l$ & Illegal\\
            \hline 0   &   0   &   0   &   0   &   0   &   0   &   0   &   0   &   0   &   1\\
                   0   &   0   &   1   &   0   &   0   &   0   &   1   &   1   &   0   &   0\\
                   0   &   1   &   0   &   0   &   1   &   1   &   0   &   0   &   0   &   0\\
                   0   &   1   &   1   &   0   &   1   &   0   &   0   &   1   &   0   &   0\\
                   1   &   0   &   0   &   1   &   0   &   0   &   0   &   0   &   1   &   0\\
                   1   &   0   &   1   &   1   &   0   &   0   &   1   &   0   &   0   &   0\\
                   1   &   1   &   0   &   0   &   0   &   1   &   0   &   0   &   1   &   0\\
                   1   &   1   &   1   &   0   &   0   &   0   &   0   &   0   &   0   &   1\\
            \end{tabular}
           	\centering
           	\caption{Wahrheitstabelle der Ansteuerung} 
            \label{abb:WahrheitstabelleAnsteuerung}
        \end{figure}
        \parindent 0pt Die Tabelle in Abbildung \ref{abb:WahrheitstabelleAnsteuerung} kann pro Signal zu folgenden logischen Verknüpfung vereinfacht werden\\
        \\
        \begin{tabular}{ccc}
            $U_h = H_1 \wedge \bar{H_2}$ & $V_h = H_2 \wedge \bar{H_3}$ & $W_h = \bar{H_1} \wedge H_3$\\
            $U_l = \bar{H_1} \wedge H_2$ & $V_l = \bar{H_2} \wedge H_3$ & $W_l = H_1 \wedge \bar{H_3}$
        \end{tabular}
